\documentclass{article}
\usepackage{fullpage,mathtools,amsfonts}
\author{Tucker DiNapoli}
\title{Numerical Analysis of the Korteweg de Vires equation}

\begin{document}
\maketitle
\section{Introduction}
A solver for the one dimensional Korteweg-de-Viers equation was written using C
and parallized using openmpi. The resultant data was plotted with gnuplot in
order to visualize the results. On inspection the plots are consistant with
results for the methods and initial conditions used. The algorithms used are
embarssingly parallel and so parallisation was trivial, and resulted in a
noticable speed up in the code.
\section{KdV Equation}
The Korteweg-de-Vires is a one dimensional partial differential
equation, of the form \[\partial_tu+\partial_x^3u-6u\partial_xu=0\]
The equation is primarily used to model waves in shallow water, and other
similar phenomena. The equation can be analytically solved via the
inverse scattering transform, or one of several numerical methods. The
stable solutions of the KdV equation take the form of solitons, waves
that move with a fixed shape.
\subsection{Initial Conditions}
In order to solve the KdV equation one must start with some sort of
initial condition defining the state of the system. In this case the
initial condition used was \[u(x)=-12\cdot csch(x)^2)\] the values of
x used were x from \(x=-1.5\pi\) to \(x=1.5\pi\) by steps of
\(\pi/32\).

\section{Numerical Analysis}
The techniques used in solving this problem were numerical
differentiation using a 5 point finite difference scheme and
Runga-Kutta forth order(rk4) time stepping. The KdV equation requires the
first and third derivatives of u with respect to x these are
calculated using the 5 point finite difference formula 
\[\frac{\mathrm{d} u_n}{\mathrm{d} x}=\frac{C_1u_{n-2}+C_2u_{n-1}+C_3u_n+C_4u_{n+1}+C_5u_{n+2}}{C_6\Delta x^{C_7}}\]
in this equation \(C_1-C_7\) constants dependent on the order of the
derivative and \(\delta x\) is the x distance between \(u_n\) \&
\(u_{n+1\). %put in actual coefficients?
Looking at the KdV equation, once the derivatives of u have been found
the time step of u can be calculated via the
equation \[\partial_tu=-\partial_x^3u+6u\partial_xu\] by using rk4
time steping.%maybe more?

\section{Code}
There are several pieces of source code that were written for this
project 

\end{document}A program for soving the Korteweg-de-Vires(KdV) partial differential equation
A one dimensional equation used (among other things) to model shallow waves,
in this case u is the height of the water above the sea floor
The equation has the form:
\[\partial_tu+\partial_x^3u-6u\partial_xu=0\]
The solver uses finite-differences in space to discretize the problem
Periodic boundary conditions are used such that the fields satisify:
u(x+Ln) = u(x,y) for all integers n; this implies the grid has a domain of
[0:L] with spacial steps \Delta x=L/mx, where mx is the number of points
the periodic boundary conditions imply that for at a boundary n where
normally n-2,n-1,n,n+1,n+2 would be used to calculate n we use
n-2,n-1,n,0,1
