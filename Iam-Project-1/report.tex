\documentclass{article}
\usepackage{fullpage}
\author{Tucker DiNapoli}
\title{Numerical Analysis of the Korteg de Viers equation}

\begin{document}
\maketitle

A program for soving the Korteweg-de-Vires partial differential equation
A one dimensional equation used (among other things) to model shallow wate
waves, where u is the hight of the water above the sea floor
The equation has the form:
\partial_tu+\partial_x^3u-6u\partial_xu=0
The solver uses finite-differences in space to discretize the problem
Periodic boundary conditions are used such that the fields satisify:
u(x+Ln) = u(x,y) for all integers n; this implies the grid has a domain of
[0:L] with spacial steps \Delta x=L/mx, where mx is the number of points
the periodic boundary conditions imply that for at a boundary n
where normally n-2,n-1,n,n+1,n+2 be used to calculate n we use
n-2,n-1,n,n-1,n-2....I think
the code will use either a hand written rk4 time step or one of several
adaptive algorithms from the gnu scientific library(i.e, QGAS,QAG..etc)
Initial conditions are taken from Zabusky \& Kruskal '65 and will be
u_0 = cos(\pix) \& u = cos(\pix-ut)
this results in a perodic solution with a recurrence time of Tr=30.4/pi
thus the default time to run is Tr.