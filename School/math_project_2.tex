\documentclass[11pt]{article}
% not necessary for xetex
% \usepackage[utf8]{inputenc}
% \usepackage[T1]{fontenc}
\usepackage{fullpage,mathtools,amsthm,enumitem,amssymb}
\usepackage{fontspec}
\setmainfont{DejaVu Serif}
\title{Math 679: Project n}
\author{Tucker DiNapoli}
\date{\today}
\newenvironment{my_proof}
{\begin{proof}\mbox{}\\*
\end{proof}}
\renewcommand \inf {\infty}
\DeclarePairedDelimiter\floor{\lfloor}{\rfloor}
\DeclarePairedDelimiter\ceil{\lceil}{\rceil}
\newcommand \Sin[1]{
  \sin\lparen#1\rparenn
}
\newcommand \Cos[1]{
  \Cos\lparen#1\rparenn
}
% \newcommand \floor[1]{
%   \lfloor #1 \rfloor
% }
% \newcommand \ceil[1]{
%   \lceil #1 \rceil
% }
\newcommand \reals {\mathbb{R}}
\newcommand \integers {\mathbb{Z}}
\newcommand \naturals {\mathbb{N}}
\newcommand \rationals {\mathbb{Q}}
%note the capital S, displays limits above/below the symbol
\newcommand \Sum[3]{
  \sum\limits_{#1}^{#2}#3
}
\begin{document}
\begin{enumerate}
\item Show \(f(x):(0,1)→\reals  = \frac{\sqrt{4 - x}-2}{x}\) has a limit at 0.
\begin{my_proof}
\end{my_proof}
\item Show \(f(x):(0,1)→\reals  = \frac{\Sin{x}}{x}\) has a limit at 0.
\begin{my_proof}
\end{my_proof}
\item Given \(f: \reals→\reals\) as a continuous function, where \(f(r) \ r^2
  ∀r∈\rationals\). Determine \(f(\sqrt{2})\) and find an expression for \(f(x),
  ∀x∈\reals\), and justify your answer.
\end{document}

%%% Local Variables:
%%% mode: latex
%%% TeX-master: t
%%% End: