% Created 2016-09-28 Wed 15:10
\documentclass[11pt]{article}
\usepackage[T1]{fontenc}
\usepackage{graphicx}
\usepackage{grffile}
\usepackage{longtable}
\usepackage{wrapfig}
\usepackage{rotating}
\usepackage[normalem]{ulem}
\usepackage{amsmath}
\usepackage{textcomp}
\usepackage{amssymb}
\usepackage{capt-of}
\usepackage{hyperref}
\usepackage{fontspec}
\usepackage{fullpage,fontspec,parskip}
\setmainfont[Ligatures=TeX]{Linux Libertine O}
\author{Tucker DiNapoli}
\date{\textit{<2016-09-28 Wed>}}
\title{Math 645 Problem Set 2}
\hypersetup{
 pdfauthor={Tucker DiNapoli},
 pdftitle={Math 645 Problem Set 2},
 pdfkeywords={},
 pdfsubject={},
 pdfcreator={Emacs 25.1.50.1 (Org mode 8.3.6)}, 
 pdflang={English}}
\begin{document}

\maketitle
\begin{enumerate}
\item \begin{enumerate}
\item \begin{center}
\begin{tabular}{llll}
 & Noodles & Pigs & Sausage\\
\hline
Noodles & 10\% & 80\% & 10\%\\
Pigs & 50\% & 20\% & 30\%\\
Sausage & 20\% & 40\% & 40\%\\
\end{tabular}
\end{center}
\item N = 0.1S + 0.8P + 0.1N\\
P = 0.5S + 0.2P + 0.3N\\
S = 0.2S + 0.4P + 0.4N\\
\item I'm not really sure what you mean by this. Even if we set S to 100 this
is an over determined system. If we plug in S = 100 and solve we get
N = P = 100, as is to be expected. The math to show this is easy enough
so I'm not going to do it out.
\end{enumerate}
\item \begin{enumerate}
\item \begin{enumerate}
\item False, Ax = 0 has the trivial solution for any A, the columns of A are
linearly independent if A has only the trivial solution
\item False, if S is a linearly dependent set, then at least one vector is a
linear combination of the others
\item True, Given a vector in R\(^{\text{4}}\) there can be at most 3 other vectors
(in ℝ⁴) that are linearly independent from it. A 4x5 matrix has 5
vectors in  ℝ⁴ meaning at least one must be a linear combination of
the others.
\item True, This is true by definition.
\end{enumerate}
\item \begin{enumerate}
\item True, If u and v are linearly independent vectors then au + bv = 0 or
u = cv, which is the equation of a line through the origin.
\item False, Any set containing the zero vector is linearly dependent, so
a set could have fewer members than dimensions but, if it had the zero
vector in it, it would still be linearly dependent.
\item True, This is true by definition.
\item False, See the answer to part b.
\end{enumerate}
\end{enumerate}
\item \begin{enumerate}
\item u,v are linearly independent vector in ℝ\(^{\text{3}}\), P is a plane through
u,v and 0, with the equation x = st + uv.
Show a linear transformation T: ℝ\(^{\text{3}}\)->ℝ\(^{\text{3}}\) maps P to either a plane,
line or point containing the origin.
\item f(x) = mx + b\\
Show that f is only a linear transformation when b == 0, why is f
called a linear function

f is linear if ∀c∈̱ℝ f(cu) = cf(u)\\
f(cu) = mcu + b, cf(u) = c*(mu + b) = cmu + c*b;\\
mcu + b = cmu + cb -> b = cb\\
b = cb only when c or b is 0, since c can be any real number in order
for f(cu) to equal cf(u) b must be 0. so f is only a linear
transformation when b is 0. It is called a linear function because it is
the function of a line (duh).

\item Show an affine transformation (T(x)=Ax + b) is a linear transformation
only when b = 0.

As with the last problem when we simplify T(cx) = cT(x) we get
A(cx) + b = A(cx) + cb -> b = cb, which is only true (for nonzero c)
when b is zero. We can make this a linear transformation by adding
another dimension (i.e using homogeneous coordinates).
\end{enumerate}

\item 1.9 \#4,6,11,15,19,23
\begin{enumerate}
\item The 2x2 rotation matrix is: [[cos(θ),-sin(θ)],[sin(θ),cos(θ)]].
So the matrix for a -π/4 radian rotation is:\\
A = [[cos(-π/4),-sin(π/4)], [sin(π/4),cos(π/4)]]\\
or [[sqrt(2)/2,-sqrt(2)/2],[sqrt(2)/2,sqrt(2)/2]].

\item the 2x2 shear matrix is [[a,1],[1,b]], where a is the horizontal
shear and b is the vertical shear. Which means to shear y by 3x
you would use the following, [[3,1],[1,0]].
\item Any reflection is just a rotation, in this case by π radians, this would
be given by the matrix:\\
A = [[cos(π),-sin(π)],[sin(π),cos(π)]] = [[-1,0],[0,-1]].
A reflection about some line is equivalent to a 3d rotation about that
line by pi radians. Despite being a 3d rotation the fact the angle is π
means the result will still be expressable as a 2d vector.
\item\relax [[3,0,-2],[4,0,0],[1,-1,1]]
\item\relax [[1,-5,4],[0,1,-6]]
\item \begin{enumerate}
\item True, the columns of an nxn identity matrix form a set of orthogonal
basis vectors in ℝⁿ, so any column vector in ℝⁿ can be expressed as a
linear combination of the columns of an identity matrix. Since T is a
linear transformation we can transform a column vector into a linear
combination of the unit vectors and then add them together and we get
the same result as if we applied T to the original vector.
\item True, a rotation of by angle φ about the origin can be expressed by
the 2x2 matrix [[cos(φ),-sin(φ)],[sin(φ),cos(φ)]]
\item False, assume f: ℝᵐ->ℝⁿ and g:ℝⁿ->ℝˡ are linear transformations, then
h = g∘f: ℝᵐ->ℝˡ must also be linear. If we let A and B be the
transformation matrices of f and g then f(x) = Ax and g(y) = By, thus
h(x) = B(Ax) = (BA)x. Since we can express h as a matrix
transformation it must be a linear transformation.
\item False, T:ℝⁿ->ℝᵐ is onto if ∀y∈ℝᵐ ∃x∈ℝⁿ such that T(x) = y. By the
wording in the question T could just map every x onto 1 y.
\item True, Assume f(x) = Ax: R³->R² was one to one, then there must exist
an inverse function g(x) = Bx: R²->R³. This would imply that we could
span R³ with only vectors from R², which is impossible, thus f cannot
be one to one.
\end{enumerate}
\end{enumerate}
\end{enumerate}
\end{document}