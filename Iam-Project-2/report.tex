\documentclass{article}
\usepackage{mathtools,fullpage,graphics,amsfonts}
\author{Tucker DiNapoli}
\title{A 2-D Parallel Finite Difference Time Domain Solver Using MPI}
\begin{document}
\maketitle
\section{Introduction}
This project aimed to develop a parallel implementation of a two
dimensional finite difference time domain(FDTD) solver implemented in C
using mpi. The primary goal of the project was the implementation of
parallization using MPI, so much of the base FDTD code was taken from
other open source FDTD solvers.%\cite{something}
The project thus consisted of two major parts, adapting the existing
code(much of which was written in C++) and parallizing the code with
MPI. 
\section{Scientific Background}
%what problem does fdtd solve
\section{Numerical Analysis}
%how how do we do fdtd
\section{Code}%maybe a better name
%quick intro
\subsection{Existing Implementations}
\subsection{Code Adaptation}
%what code did I adapt and what did I do
\subsection{Parallization}
%what did I do generically(ie what parts did I split up)
%where did i need to transfer info etc
%Then How did I do this via mpi
\section{Data}
%hopefully show an example use of my code, try and do some profiling
%Also do some examples of how does the parallization speed up the code
\section{Conclusion}
\bibliographystyle{ieeetr}
\bibliography{E-M-Project}
\end{document}
