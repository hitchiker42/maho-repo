% Created 2016-02-03 Wed 11:37
\documentclass[11pt]{article}
\usepackage[utf8]{inputenc}
\usepackage[T1]{fontenc}
\usepackage{fixltx2e}
\usepackage{graphicx}
\usepackage{grffile}
\usepackage{longtable}
\usepackage{wrapfig}
\usepackage{rotating}
\usepackage[normalem]{ulem}
\usepackage{amsmath}
\usepackage{textcomp}
\usepackage{amssymb}
\usepackage{capt-of}
\usepackage{hyperref}
\author{Tucker DiNapoli}
\date{\today}
\title{}
\hypersetup{
 pdfauthor={Tucker DiNapoli},
 pdftitle={},
 pdfkeywords={},
 pdfsubject={},
 pdfcreator={Emacs 25.0.50.1 (Org mode 8.3.2)}, 
 pdflang={English}}
\begin{document}

\tableofcontents

As far as I know all of my code works. I also included my implementations of
merge sort and heap sort in my code. I wrote two versions of radix sort, one
that uses 8 bit (byte) histograms and one that uses 16 bit (word)
historgrams. The radix sort on the included plot is the one that uses 8 bit
histograms, I would have included one for the 16 bit version, but as I write
this I don't have time to re-run all the tests to generate the graph. The byte
version performs 9 loops over the data plus a loop of length 256 over the
histograms, while the word version performs 5 loops over the data plus a loop
of length 65536 over the histograms.The performance of the two is what you
would expect based on the loop count, for arrays up to about 100,000 elments
the byte version is faster, due to having smaller constant factors (i.e the
loop over the histograms), beyond that the word verson is faster. For arrays of
a million or more the word version is consistantly about twice as fast as the
byte version, which is to be expected. Both of my radix sorts sort 64 bits,
regardless of the number of bytes used by the input, it wouldn't be that hard
to optimize this so I skipped the later loops if the input used fewer than 64
bits, but I didn't do that (I guess because I'm lazy?).


The performances of the algorithms are mostly what you would expect. My version
of quick sort is faster than the system quicksort by some factor of n, due to
the fact the system quick sort has the added overhead of a function call for
each comparision. Both quick sorts are slower than radix sort, though my
quick sort is just barely slower, which is somewhat suprising, though I imagine
you would see more of a difference if I used the word version of radix sort, or
optimized my radix sort by the number of bytes in the input, as it is it runs
at O(8*n) as it is now. The counting sort is the fastest since it runs at O(n)
for the 16 bit integers, also my counting sort is inplace which elimnates some
of the overhead of radix sort.

In my personal testing I found, for comparision based sorts on random data,
that quicksort is usually the fastest, followed by merge sort, followed by
heapsort. Heapsort is typically significantly slower, likely because I use
insertion sort for sorting the small sublists in quick/merge sort.

For fun I also did a comparision of bubblesort and insertion sort, I've never
actually used bubble sort before, and it really is as bad as it's reputation,
insertion sort is typically about an order of magnitude faster.




\begin{verbatim}
#define NIL -1
int search(int *A, int n, int v){
  int i;
  for(i=0;i<n;i++){
    if(A[i] == v){
      return i;
    }
  }
  return NIL;
}
\end{verbatim}
Loop Invariant: For all 0<=j<i A[j] != v;\\
Initialization: i = 0, so \{j: 0<=j<i\} is empty\\
Maintenance: If there exists some 0<=j<i such that A[j] = v the loop
would have terminated when i was equal to j, thus A[j] != v for all j<=0<i
Termination: At end of the loop i = n, so for all 0<=j<n A[j] != v, thus\\
v is not in A, so we can return NIL

\begin{verbatim}
Let h(n) = f(n) + g(n);
We can assume max(f(n),g(n)) = f(n), without a loss of generality.
In this case ∃N such that ∀n ≥ N f(n) ≥ g(n), or equivlemently f(n) ≥ 0.5*h(n).
Since h(n) = f(n) + g(n) we can also say 2*h(n) ≥ f(n).
These two inequalities together imply ∃N 2*h(n) ≥ 2*f(n) ≥ 0.5*h(n) ∀n ≥ N, meaning
f(n) = Θ(h(n)), or max(f(n),g(n)) = Θ(f(n)+g(n))∎
\end{verbatim}
\end{document}